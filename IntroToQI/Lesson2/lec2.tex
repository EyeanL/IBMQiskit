\documentclass[11pt]{scrartcl}
\usepackage{graphicx}
\graphicspath{{images/}}
\usepackage[usenames,dvipsnames,svgnames]{xcolor}
\usepackage[shortlabels]{enumitem}
\usepackage[framemethod=TikZ]{mdframed}
\usepackage{amsmath,amssymb,amsthm}
\usepackage{epigraph}
\usepackage{hyperref}
\usepackage{tikz}
\usepackage{microtype}
\usepackage[headsepline]{scrlayer-scrpage}
\usepackage{thmtools}
\usepackage{amsmath}
\usepackage{amssymb}
\usepackage{braket}
\renewcommand{\epigraphsize}{\scriptsize}
\renewcommand{\epigraphwidth}{60ex}
\addtolength{\textheight}{3.14cm}
\ihead{\footnotesize\textbf{Lesson 2 - Multiple Systems}}
\ohead{\footnotesize IBM: Introduction to Quantum Computing}
\providecommand{\clubs}[1]{$#1\clubsuit$}
\providecommand{\ol}{\overline}
\providecommand{\eps}{\varepsilon}
\providecommand{\half}{\frac{1}{2}}
\providecommand{\dang}{\measuredangle} %% Directed angle
\providecommand{\CC}{\mathbb C}
\providecommand{\FF}{\mathbb F}
\providecommand{\NN}{\mathbb N}
\providecommand{\QQ}{\mathbb Q}
\providecommand{\RR}{\mathbb R}
\providecommand{\ZZ}{\mathbb Z}
\providecommand{\dg}{^\circ}
\providecommand{\ii}{\item}
\reversemarginpar
\providecommand{\printpuid}[1]{\marginpar{\ttfamily\footnotesize\color{green!40!black}#1}}
\usepackage{multicol}


\mdfdefinestyle{mdbluebox}{roundcorner=10pt,innerbottommargin=9pt,
	linecolor=blue,backgroundcolor=TealBlue!5,}
\declaretheoremstyle[headfont=\sffamily\bfseries\color{MidnightBlue},
	mdframed={style=mdbluebox},]{thmbluebox}
\mdfdefinestyle{mdredbox}{frametitlefont=\bfseries,innerbottommargin=8pt,
	nobreak=true,backgroundcolor=Salmon!5,linecolor=RawSienna,}
\declaretheoremstyle[headfont=\bfseries\color{RawSienna},
	mdframed={style=mdredbox},headpunct={\\[3pt]},postheadspace=0pt,]{thmredbox}
\mdfdefinestyle{mdgreenbox}{linecolor=ForestGreen,backgroundcolor=ForestGreen!5,
	linewidth=2pt,rightline=false,leftline=true,topline=false,bottomline=false,}
\declaretheoremstyle[headfont=\bfseries\sffamily\color{ForestGreen!70!black},
	mdframed={style=mdgreenbox},headpunct={ --- },]{thmgreenbox}
\mdfdefinestyle{mdblackbox}{linecolor=black,backgroundcolor=RedViolet!5!gray!5,
	linewidth=3pt,rightline=false,leftline=true,topline=false,bottomline=false,}
\declaretheoremstyle[mdframed={style=mdblackbox}]{thmblackbox}
\declaretheorem[style=thmbluebox,name=Problem]{problem}
\declaretheorem[style=thmbluebox,name=Theorem,numberwithin=problem]{theorem}
\declaretheorem[style=thmbluebox,name=Lemma,sibling=theorem]{lemma}
\declaretheorem[style=thmbluebox,name=Theorem,numbered=no]{theorem*}
\declaretheorem[style=thmbluebox,name=Lemma,numbered=no]{lemma*}
\declaretheorem[style=thmgreenbox,name=Claim,sibling=theorem]{claim}
\declaretheorem[style=thmgreenbox,name=Claim,numbered=no]{claim*}
\declaretheorem[style=thmblackbox,name=Remark,sibling=theorem]{remark}
\declaretheorem[style=thmblackbox,name=Remark,numbered=no]{remark*}
\declaretheorem[style=thmgreenbox,name=Definition,sibling=theorem]{definition}
\declaretheorem[style=thmgreenbox,name=Definition,numbered=no]{definition*}
\newcommand{\gfigure}[4]
{
	\begin{figure}
      \begin{center}
          \includegraphics[#3]{#4}
          \caption{#1}
          \label{Figure:#2}
      \end{center}s
    \end{figure}
}
\newcommand{\figref}[1]{Figure~\ref{Figure:#1}}

\title{L}
\author{Ian Lu}
\date{Oct 17 2023}

\begin{document}
\title{}
\section{Multiple Systems}
The general idea is to view multiple systems together as a formation of a single, compound system. This concept 
leads to a description of how quantum states, measurements, and operations work for multiple systems.
\subsection{Classical Information}
\subsubsection{Classical States via the Cartesian Product}
For simplicity, lets first begin by introducing two systems then generalizing to more than two systems.
Let X be a system whose classical state set is $\Sigma$ and Y be a second system with classical state set $\Gamma$.
\vspace{5mm}\\
Reasonably assume that $\Sigma$ and $\Gamma$ are two finite and nonempty sets that do not necessarily equal one another.
Now imagine the two systems X and Y, placed side-by-side to one another, with X on the left and Y on the right. 
\vspace{5mm}\\We can view the two systems
as a single system denoted by (X, Y) or simply XY. The set of classical states of (X, Y) is the \textit{Cartesian Product} of 
$\Sigma$ and $\Gamma$, defined as:
$$\Sigma \times \Gamma = \{(a, b) \; : \; a\in\Sigma \; \textbf{and} \; b\in\Gamma \}.$$
Essentially, the Cartesian Product is the mathematical notion that captures the idea of viewing an element of on set and an element of another set together to 
form a single element of a single (combined) set. 
To say (X, Y) is in the classical state $(a, b)\in\Sigma \times \Gamma$ means that X is in the classical state $a\in\Sigma$ and Y is in the classical state
$b\in\Gamma$. The classical state of the joint system (X, Y) is (a, b).
\vspace{5mm}\\For more than two systems we can generalize classical states of the joint systems. Suppose $X_1, \dots ,X_n$ are systems with classical state sets
$\Sigma_1, \dots ,\Sigma_n$, the cartesian product of the compound system is: $$\Sigma_1 \times \dots \times \Sigma_n = \{ (\sigma_1, \dots, \sigma_n) : \sigma_1 \in \Sigma_1, \dots, \sigma_n \in \Sigma_n \}.
$$ \vspace{-5mm}
\vspace{5mm}\\For example if $\Sigma_1 = \Sigma_2 = \Sigma_3 = \{0, 1\}$ (a system of 3 bits), then the classical state 
set of $(X_1, X_2, X_3)$ is: 
$$\Sigma_1 \times \Sigma_2 \times \Sigma_3 = \{ (0,0,0), (0,0,1), (0,1,0), (0,1,1), (1,0,0), (1,0,1), (1,1,0), (1,1,1) \}$$
\subsubsection{Representing States as Strings}
It is convenient to write a classical state. $(a_1, \dots ,a_n)$ as a string $a_1, \cdots ,a_n$.
Consider a 10 bit system $(X_1, \dots ,X_n)$ with $2^{10} = 1024$ classical states which are the elements of the set:
$$\Sigma_1 \times \dots \times \Sigma_{10} = \{0, 1\}^{10}$$The cartesian product is ordered lexicographically (significance of index decreases left to right), where
the previous system is represented in strings as such:
\begin{center}
	0000000000 \\
	0000000001 \\
	0000000010 \\
	0000000100 \\
	$\vdots$ \\
	1111111111
\end{center}
\end{document}